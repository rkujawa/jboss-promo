%
% Copyright (c) 2013 Radoslaw Kujawa.
% All rights reserved.
%
% Redistribution and use in source and binary forms, with or without
% modification, are permitted provided that the following conditions
% are met:
%
% 1. Redistributions of source code must retain the above copyright
%    notice, this list of conditions and the following disclaimer.
% 2. Redistributions in binary form must reproduce the above copyright
%    notice, this list of conditions and the following disclaimer in the
%    documentation and/or other materials provided with the distribution.
%
% THIS SOFTWARE IS PROVIDED BY RADOSLAW KUJAWA (THE AUTHOR) AND CONTRIBUTORS
% ``AS IS'' AND ANY EXPRESS OR IMPLIED WARRANTIES, INCLUDING, BUT NOT LIMITED
% TO, THE IMPLIED WARRANTIES OF MERCHANTABILITY AND FITNESS FOR A PARTICULAR
% PURPOSE ARE DISCLAIMED.  IN NO EVENT SHALL THE AUTHOR OR CONTRIBUTORS
% BE LIABLE FOR ANY DIRECT, INDIRECT, INCIDENTAL, SPECIAL, EXEMPLARY, OR
% CONSEQUENTIAL DAMAGES (INCLUDING, BUT NOT LIMITED TO, PROCUREMENT OF
% SUBSTITUTE GOODS OR SERVICES; LOSS OF USE, DATA, OR PROFITS; OR BUSINESS
% INTERRUPTION) HOWEVER CAUSED AND ON ANY THEORY OF LIABILITY, WHETHER IN
% CONTRACT, STRICT LIABILITY, OR TORT (INCLUDING NEGLIGENCE OR OTHERWISE)
% ARISING IN ANY WAY OUT OF THE USE OF THIS SOFTWARE, EVEN IF ADVISED OF THE
% POSSIBILITY OF SUCH DAMAGE.
%
% 
\documentclass[dvipsnames,table]{beamer}
\usepackage{polski}

\usetheme{Rochester}
\usecolortheme{orchid}

\usepackage{listings}
\usepackage{ucs}
\usepackage[utf8x]{inputenc}
\usepackage{wasysym}
\usepackage[normalem]{ulem}
\usepackage{amsmath}
\usepackage{hyperref}

\setbeamertemplate{navigation symbols}{}
\setbeamertemplate{caption}[numbered]
\setbeamerfont{caption}{size=\scriptsize}
\setbeamercolor{framenote}{bg=OSEC-red!25}
\setbeamercolor{rednote}{bg=Red!25}
\setbeamercolor{palette primary}{use=structure,fg=white,bg=OSEC-red}
\setbeamercolor{palette secondary}{use=structure,fg=white,bg=OSEC-red2}

\setbeamertemplate{itemize item}{\scriptsize\raise1pt\hbox{\donotcoloroutermaths$\blacktriangleright$}}
\setbeamertemplate{itemize subitem}{\tiny\raise1pt\hbox{\donotcoloroutermaths$\bullet$}}
\setbeamertemplate{itemize subsubitem}{\tiny\raise1pt\hbox{\donotcoloroutermaths{--}}}

\setbeamertemplate{enumerate item}{\insertenumlabel.}
\setbeamertemplate{enumerate subitem}{\insertenumlabel.\insertsubenumlabel}
\setbeamertemplate{enumerate subsubitem}{\insertenumlabel.\insertsubenumlabel.\insertsubsubenumlabel}
\setbeamertemplate{enumerate mini template}{\insertenumlabel}

\setbeamercolor{itemize item}{fg=OSEC-red, bg=OSEC-red}
\setbeamercolor{itemize subitem}{fg=OSEC-red, bg=OSEC-red}
\setbeamercolor{itemize subsubitem}{fg=OSEC-red, bg=OSEC-red}

\setbeamercolor{section number projected}{fg=white,bg=OSEC-red}
\setbeamercolor{subsection number projected}{fg=white,bg=OSEC-red}
\setbeamercolor{button}{bg=OSEC-red,fg=white}

\setbeamertemplate{section in toc}[circle]
\setbeamertemplate{subsection in toc}[square]


\definecolor{OSEC-red}{RGB}{160,29,44}
\definecolor{OSEC-red2}{RGB}{177,76,12}
\hypersetup{colorlinks=true,linkcolor=white,urlcolor=OSEC-red}

\setlength{\tabcolsep}{8pt}
\renewcommand{\arraystretch}{1.2}

\newcommand{\nbsdcolor}[1] {
	{\color{OSEC-red} #1}
}

\lstset{
   language=sh,
   basicstyle=\tiny,
   breaklines=true,
   escapechar=\@,
   commentstyle=\color{OSEC-red}
}

\AtBeginSection[]{
\frame{

\begin{center}

{\usebeamerfont{section name}\usebeamercolor[fg]{section name}Część~\insertsectionnumber}
    \vskip1em\par

	\begin{beamercolorbox}[sep=12pt,center]{palette primary}
		\usebeamerfont{section title}\insertsection\par
	\end{beamercolorbox}
\end{center}

%\sectionpage
}
}

\title{Nowości technologiczne w \\ JBoss Enterprise Application Platform 6: \\ zarządzanie w dużych ogranizacjach}


\author{Radosław Kujawa -- radoslaw.kujawa@osec.pl}

\institute{OSEC}

\begin{document}

\begin{frame}
\titlepage
\end{frame}

\begin{frame}[allowframebreaks]
\frametitle{Spis treści}
{
\hypersetup{colorlinks=true,linkcolor=black,urlcolor=OSEC-red}
\tableofcontents
}
\end{frame}


\section{JBoss EAP a zarządzanie IT w dużej organizacji}

\begin{frame}
\frametitle{Zarządzanie JBoss EAP w dużych środowiskach}
\begin{itemize}
	\item Problemy w poprzedniej wersji EAP (5.x)
	\begin{itemize}
		\item Nawet wykonanie prostych czynności wymagało specjalistycznej wiedzy z zakresu działania serwera aplikacyjnego
		\item Konsola web praktycznie bezużyteczna dla celów administracji -- konieczność ręcznej edycji plików XML
		\item Duży stopień skomplikowania konfiguracji, wiele niejawnych zależności między komponentami
		\item Przy większej skali wdrożenia zmiany w konfiguracji trudne do zaaplikowania
	\end{itemize}
\end{itemize}
\end{frame}

\begin{frame}
\frametitle{Zarządzanie JBoss EAP w dużych środowiskach}
\begin{itemize}
	\item Nowe wydanie JBoss EAP 6 jest oparte o JBoss AS 7 (teraz ,,Wildfly'')
	\begin{itemize}
		\item Znacznie silniejsza integracja z Red Hat Enterprise Linux -- jednolite narzędzia do zarządzania i wysoki poziom bezpieczeństwa
		\item Mechanizmy pozwalające na wprowadzanie zmian w serwerze w sposób zautomatyzowany, nawet w środowiskach dużej skali
	\end{itemize}
	\item Próba zmiany sposobu myślenia o serwerze aplikacyjnym
	\begin{itemize}
		\item Nowe podejście: traktujmy platformę składającą się z serwera aplikacyjnego i systemu operacyjnego jako całość
	\end{itemize}
\end{itemize}
\end{frame}




\section{Metodyka instalacji i aktualizacji}

\begin{frame}
\frametitle{Instalacja i aktualizacja JBoss EAP z użyciem yum/rpm}
\begin{itemize}
	\item Od wersji 6 preferowaną metodą instalacji i aktualizacji jest użycie tych samych mechanizmów {\tt yum} oraz {\tt rpm}, które używane są do zarządzania oprogramowaniem w RHEL\footnote{Instalacja dalej jest możliwa też na ,,stary'' sposób - poprzez rozpakowanie zip'a lub użycie graficznego instalatora.}.
	\item Trywialna instalacja
	\begin{itemize}
		\item Włączenie kanału JBoss EAP 6 dla danej maszyny w Red Hat Network
		\item Na serwerach aplikacyjnych: {\tt yum groupinstall jboss-eap6}
		\item Na maszynach frontend: {\tt yum install mod\_jk-ap22 mod\_cluster-native}
	\end{itemize}
	\item Rozwiązywanie zależności między paczkami
	\item Łatwe aktualizacje systemu operacyjnego oraz serwera aplikacji
	\begin{itemize}
		\item {\tt yum update \&\& service jboss-xxx restart}
	\end{itemize}

\end{itemize}
\end{frame}

\begin{frame}
\frametitle{Instalacja i aktualizacja JBoss EAP z użyciem yum/rpm - c.d.}
\begin{itemize}
	\item Struktura katalogów przyjazna dla administratora systemu operacyjnego
	\begin{itemize}
		\item Logi w {\tt /var/log}, konfiguracja w {\tt /etc}, itd.
		\item ,,Stara'' struktura dostępna przez {\tt /usr/share/jbossas}.
	\end{itemize}
	\item Osobna usługa systemu operacyjnego dla każdej instancji JBossa.
	\item Domyślne przykładowe konfiguracje dla trybów osadzenia {\tt standalone} oraz {\tt domain}.
\end{itemize}
\end{frame}

\begin{frame}
\frametitle{Instalacja i aktualizacja JBoss EAP z użyciem yum/rpm - c.d.}
\begin{itemize}
	\item Użycie {\tt yum} i {\tt rpm} powoduje, że do zarządzania instalacjami JBoss można użych tych samych narzędzi co do zarządzania systemem.
	\begin{itemize}
		\item Red Hat Network / Red Hat Satellite / SpaceWalk
	\end{itemize}
	\item Łatwa weryfikacja poziomu ,,dopatchowania''
	\item Które JBossy wymagają instalacji poprawek?
	\item Uniknięcie problemu wdrożonych serwerów aplikacyjnych, które nigdy później nie są aktualizowane i stanowią zagrożenie bezpieczeństwa.
\end{itemize}
\end{frame}

\section{Nowe API do zarządzania serwerami aplikacyjnymi}

\begin{frame}
\frametitle{Nowe API do zarządzania platformą}
\begin{itemize}
	\item Całkowicie przepisane w porównaniu z EAP5, zastępuje dotychczasowe API JMX
	\item Oparte o komunikaty JSON
	\item ,,Zdetypowane'' -- zmiany możliwe z innej maszyny nawet bez natywnych bibliotek serwera
	\item Modyfikacja wszystkich elementów konfiguracji serwera możliwa za pomocą API
	\begin{itemize}
		\item Klient linii komend dający pełny dostęp do API
		\item Zmiany za pomocą API skutkują zmianami w plikach XML
		\item W końcu możliwe bezpieczne wykonywanie zmian z konsoli web (konsola też jest klientem API)
		\item Biblioteki klienckie do budowy własnych skryptów
	\end{itemize}
	\item Statystyki pracy serwera dostępne przez API
\end{itemize}
\end{frame}

\begin{frame}
\frametitle{Zalety nowego API}
\begin{itemize}
	\item Skryptowalne operacje związane z zarządzaniem aplikacjami na serwerze: osadzenie nowej aplikacji (także zdalnie), zatrzymanie i usunięcie aplikacji
	\item Rolling deployment za pomocą tzw. deployment planów
	\item Konfiguracja konektorów baz danych
	\item Łatwiejsze tworzenie testów integracyjnych (zmiany konfiguracji serwera z poziomu maven'a, osadzanie aplikacji np. za pomocą plugina CARGO)
	\item Możliwość tworzenia wtyczek do monitoringu dla oprogramowania producentów trzecich
\end{itemize}
\end{frame}

\begin{frame}
\frametitle{Przykładowe operacje możliwe do wykonania za pomocą API}
\begin{itemize}
	\item Kod
\end{itemize}
\end{frame}


\begin{frame}
\frametitle{Inne funkcjonalności godne wspomnienia}
\begin{itemize}
	\item Uproszczone zarządzanie większą liczbą serwerów o takiej samej konfiguracji dzięki domenom oraz profilom.
	\item Bar \textbf{xyz}
\end{itemize}
\end{frame}

\begin{frame}
\frametitle{Koniec\ldots}
%\vspace*{0.15cm}
\begin{center}
\includegraphics[scale=0.5]{img-oseclogo.png}

Dziękuje!

Czy są pytania?
\end{center}

\end{frame}

 
\end{document}
